% !TEX encoding = UTF-8
% !TeX spellcheck = en_US
% !TEX root = ICPS_manipulator.tex
% !TEX program = pdflatex
% !TEX options = -shell-escape
\documentclass[conference]{IEEEtran}
\IEEEoverridecommandlockouts
% The preceding line is only needed to identify funding in the first footnote. If that is unneeded, please comment it out.
\usepackage{cite}
\usepackage{amsmath,amssymb,amsfonts}
\usepackage{algorithmic}
\usepackage{graphicx}
\usepackage{textcomp}
\usepackage{xcolor}
\def\BibTeX{{\rm B\kern-.05em{\sc i\kern-.025em b}\kern-.08em
    T\kern-.1667em\lower.7ex\hbox{E}\kern-.125emX}}

% fix encoding warning
\usepackage[OT1]{fontenc}

% hyperlinks
\usepackage{hyperref}
\hypersetup{
    colorlinks = true,
    citecolor = red,
    linkcolor = blue,
    filecolor = magenta,
    urlcolor = cyan,
}

% definitions
\newtheorem{assumption}{Assumption}
\newtheorem{definition}{Definition}
\newtheorem{lemma}{Lemma}
\newtheorem{remark}{Remark}
\newtheorem{example}{Example}
\newtheorem{corollary}{Corollary}
\newtheorem{proposition}{Proposition}

% Math defenitions
\DeclareMathOperator{\R}{\mathbb{R}}

% TIKZ
\usepackage{tikz}
\usepackage{import}

\begin{document}

\title{
    The kinematic calibration of industrial manipulators using Force/Torque sensor
    % \thanks{Identify applicable funding agency here. If none, delete this.}
}

\author{
    \IEEEauthorblockN{Anna Trufanova*, Alexey Ovcharov, Vladislav Antipov, Cherginets Dmitriy, Alexey Vedyakov}
    \IEEEauthorblockA{
        Faculty of Control Systems and Robotics \\
        Saint Petersburg National Research University of Information Technologies, Mechanics and Optics (ITMO University) \\
        Saint Petersburg, Russia \\
        Email: a_trufanova@itmo.ru
    }
}

\maketitle

\begin{abstract}
    This document is a model and instructions for \LaTeX.
    This and the IEEEtran.cls file define the components of your paper [title, text, heads, etc.]. *CRITICAL: Do Not Use Symbols, Special Characters, Footnotes,
or Math in Paper Title or Abstract.
\end{abstract}

\begin{IEEEkeywords}
    component, formatting, style, styling, insert
\end{IEEEkeywords}

\section{Introduction}
The development of the industry does not stop for a day. And every year more and more different tasks are automatization and most of the operations in them need an accurate position control of the tool. Some of them require high accuracy. For example, use a manipulator for milling or drilling in production is very attractive due to its versatility, large work area, and relatively low cost, but it is not enough to use the kinematic parameters from the technical manual to obtain the required product quality.\\


The accuracy of the tool positioning is defined by the accuracy of the kinematical model parameters, which used in forward and inverse kinematics tasks. Inaccuracies in kinematic parameters may arise due to the impossibility of accurately producing manipulators part the defined size and due to assembly error in the assembly process.\\

% Add: Исходя из этого есть необходимость в корректировке кинематических параметров. Что-то написать про то что многие геометрические способы включающае в себя очень точные датчики слишком дорогие, а другие геометрические методыподразумевают использование человека в процессе снятия опорных точек. Но индустрия 4.0 подразумевает переход к автоматизации, а значит и из этой задачи нужно избавляться от человеческого рабского труда. И метод представленный в этой статье как раз и позволяет это сделать.

A lot of models are available in the literature for kinematic modeling of robotic manipulators as Classic Denavit-Hartenberg model (DH) and modified DH model,  a complete and parametrically continuous kinematic model, the product of exponential model, and S model.\\


\section{Model and problem formulation}
% !TEX encoding = UTF-8
% !TeX spellcheck = en_US
% !TEX root = ../ICPS_manipulator.tex
The only four unique DH parameters $\theta_i$, $d_i$, $a_i$, $\alpha_i \in \R$ for each $i$ joint are commonly used to describe the serial chain manipulator kinematic \cite{Spong}. The transform matrix $T_{i-1}^i$ contains this parameters and describes transformation from the frame $i-1$ to $i$. The forward kinematic solution gives us the end-effector (frame $n$) transformation relative the base of manipulator frame $0$, using multiplication of transform matrices $i = \overline{1, n}$ we get this solution
\begin{align}
    T_{i - 1}^i & = \begin{bmatrix}
        R_{i-1}^{i} & o_{i_1}^{i} \\
        0 & 1
    \end{bmatrix} =
    \begin{bmatrix}
        c_{\theta_i} & -s_{\theta_i}c_{\alpha_i} & s_{\theta_i}s_{\alpha_i} & a_{i}c_{\theta_i} \\
        s_{\theta_i} & c_{\theta_i}c_{\alpha_i} & -c_{\theta_i}s_{\alpha_i} & a_{i}s_{\theta_i} \\
        0 & s_{\alpha_i} & c_{\alpha_i} & d_i \\
        0 & 0 & 0 & 1
    \end{bmatrix}, \nonumber \\
    \label{eq:fk-solution}
    T_0^n & = \prod_{i=1}^n T_{i - 1}^i,
\end{align}
where $c_{(\cdot)} = \cos(\cdot)$ and $s_{(\cdot)} = \sin(\cdot)$, see Fig. \ref{fig:dh-convension}.
\begin{figure}[h!]
    \centering
    \import{img/tikz}{DH_Frames.pdf_tex}
    \label{fig:dh-convension}
    \caption{DH convention visualization}
\end{figure}

Let us consider the revolute joint serial manipulator with Force/Torque sensor (Fig. \ref{fig:eef-scheme}). Relation between force and joint torques
\begin{align}
    \label{eq:ft-relation}
    \tau & = J^T \mathcal{F}, &
    \mathcal{F} = \begin{bmatrix} F_e \\ \tau_e \end{bmatrix},
\end{align}
where $\tau_e, F_e \in \R^3$ is the torque and force applied to end-effector that combined in the generalized force vector $\mathcal{F}$, $\tau \in \R^n$ in the joint torque vector and $J \in \R^{6 \times 6}$ is manipulator Jacobian.
\begin{figure}[h!]
    % \hspace*{2cm}
    \centering
    \import{img/tikz}{Manipulator_with_FTsensor.pdf_tex}
    \label{fig:eef-scheme}
    \caption{End-effector scheme}
\end{figure}

In this work we solve the \textbf{kinematic model DH parameters estimation problem}. Consider the serial manipulator with revolute joint, the forward kinematic solution \eqref{eq:fk-solution} and the force - torque relation \eqref{eq:ft-relation}. Define the unknown parameters vector $p_i$ for each joint $i = \overline{1, n}$ that combined in the vector $P$
\begin{align}
    p_i & =
    \begin{bmatrix}
        d_i \\ a_i
    \end{bmatrix}, &
    P & =
    \begin{bmatrix}
        p_0 \\ p_1 \\ \vdots \\ p_n
    \end{bmatrix}, &
    i & = \overline{1, n}.
\end{align} \\
Design the estimator
\begin{align}
\hat{P} = f(\mathcal{F}, \tau, \theta, \alpha)
\end{align}
such that
\begin{align}
    \lim_{t\to\infty} {P - \hat{P}} = 0
\end{align}

As usual to design the estimator we need the following assumptions
\begin{assumption}
    The only measurable signals are $\mathcal{F}$, $\tau$, $\theta$, the $\alpha$ parameter is known
\end{assumption}

{\color{red} **Problem of the section above}

{\color{red} Say that we estimate only $d_i$ and $a_i$}

{\color{red} Define the vectors $\theta = \begin{bmatrix} \theta_1 \\ \theta_2 \\ \vdots \\ \theta_n \end{bmatrix}$, $\alpha = \begin{bmatrix} \alpha_1 \\ \alpha_2 \\ \vdots \\ \alpha_n \end{bmatrix}$}


\section{Kinematic parameters estimation}
This document is a model and instructions for \LaTeX.
Please observe the conference page limits.


\section{Simulation and experimental results}
This document is a model and instructions for \LaTeX.
Please observe the conference page limits.

\section{Conclusion and future work}
This document is a model and instructions for \LaTeX.
Please observe the conference page limits.

\end{document}
