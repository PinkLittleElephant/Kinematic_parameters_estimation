The development of the industry does not stop for a day. And every year more and more different tasks are automatization and most of the operations in them need an accurate position control of the tool. Some of them require high accuracy. For example, use a manipulator for milling or drilling in production is very attractive due to its versatility, large work area, and relatively low cost, but it is not enough to use the kinematic parameters from the technical manual to obtain the required product quality.\\


The accuracy of the tool positioning is defined by the accuracy of the kinematical model parameters, which used in forward and inverse kinematics tasks. Inaccuracies in kinematic parameters may arise due to the impossibility of accurately producing manipulators part the defined size and due to assembly error in the assembly process.\\

% Add: Исходя из этого есть необходимость в корректировке кинематических параметров. Что-то написать про то что многие геометрические способы включающае в себя очень точные датчики слишком дорогие, а другие геометрические методыподразумевают использование человека в процессе снятия опорных точек. Но индустрия 4.0 подразумевает переход к автоматизации, а значит и из этой задачи нужно избавляться от человеческого рабского труда. И метод представленный в этой статье как раз и позволяет это сделать.

A lot of models are available in the literature for kinematic modeling of robotic manipulators as Classic Denavit-Hartenberg model (DH) and modified DH model,  a complete and parametrically continuous kinematic model, the product of exponential model, and S model.\\
